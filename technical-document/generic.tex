%%
%% Copyright (C) 2015 Jihuan Tian <jihuan_tian@hotmail.com>
%%  
%% This program is free software; you can redistribute it and/or modify
%% it under the terms of the GNU General Public License as published by
%% the Free Software Foundation; either version 2 of the License, or
%% (at your option) any later version.
%%  
%% This program is distributed in the hope that it will be useful,
%% but WITHOUT ANY WARRANTY; without even the implied warranty of
%% MERCHANTABILITY or FITNESS FOR A PARTICULAR PURPOSE.  See the
%% GNU General Public License for more details.
%%  
%% You should have received a copy of the GNU General Public License
%% along with this program; if not, write to the Free Software
%% Foundation, Inc., 59 Temple Place - Suite 330, Boston, MA 02111-1307, USA.
%%  

\usepackage{ifthen}

\makeatletter
\@ifclassloaded{beamer}{
  %% Configuration for the beamer class.
}{
  %% Configuration for other classes.
  
  \ifthenelse{\(\isundefined\pdfoutput\)\OR\(\pdfoutput < 1\)}{
    %% When the PDF output is not enabled.
    \usepackage[dvips]{graphicx}
    \usepackage[dvipdfm, bookmarksopen=true, bookmarksnumbered=true, colorlinks=true, citecolor=blue, hyperindex=true, plainpages=false, pdfpagelabels=true, destlabel=true]{hyperref}
  }{
    %% When the PDF output is enabled.
    \usepackage[pdftex]{graphicx}
    %% For subfigure typesetting
    \usepackage{subfigure}

    \usepackage[pdftex, unicode=true, bookmarksopen=true, bookmarksnumbered=true, colorlinks=true, citecolor=blue, hyperindex=true, plainpages=false, pdfpagelabels=true, destlabel=true]{hyperref}
  }

  \hypersetup{
    pdfnewwindow=true
  }

  %% Use color by names
  \usepackage[usenames]{xcolor}

  % Let "paragraph" be the 4-th level heading.
  \usepackage{titlesec}
  \setcounter{secnumdepth}{4}
  \setcounter{tocdepth}{4}
  \titleformat{\paragraph}
  {\normalfont\normalsize\bfseries}{\theparagraph}{1em}{}
  \titlespacing*{\paragraph}
  {0pt}{3.25ex plus 1ex minus .2ex}{1.5ex plus .2ex}
}
\makeatother

%% Customized colors
\definecolor{comment-green}{RGB}{0, 102, 0}
\definecolor{solarized-yellow}{RGB}{246, 240, 222}
\definecolor{code-hl}{RGB}{204, 255, 255}

% Highlight color
\usepackage{soul}
\sethlcolor{code-hl}

% SI units
\usepackage[amssymb]{SIunits}

% Typesetting chemistry
\usepackage{mhchem}

% For typesetting the tilde symbol in normal text.
\usepackage{textcomp}

% Typesetting source code
\usepackage{listings}
\lstset{
  basicstyle=\ttfamily,
  keywordstyle=\color{blue}\bfseries,
  commentstyle=\color{comment-green},
  stringstyle=\color{magenta},
  columns=fullflexible,
  frame=single,
  breaklines=true,
  postbreak=\mbox{\textcolor{red}{$\hookrightarrow$}\space},
}
% Define an empty language which is responsible for typesetting text-mode source
% code block in Emacs.
\lstdefinelanguage{Text}{}

% in the document preamble
 \usepackage[stable]{footmisc}

% Bibliography
\makeatletter
\@ifclassloaded{beamer}{
  \bibliographystyle{alpha}
}{
  \usepackage[round, authoryear]{natbib}
  \bibliographystyle{plainnat}}
\makeatother

% Several symbols used for writing draft
\usepackage{bbding}
\newcommand{\outline}[1]{\noindent \textcolor{blue}{\NibSolidRight\;#1}}
\usepackage{fontawesome}
\newcommand{\draft}[1]{\noindent \textcolor{comment-green}{\faCommenting\;#1}}
\newcommand{\todo}[1]{\noindent \textcolor{red}{\faCalendar\;#1}}

% Four kinds of emphasis style
\newcommand{\emphr}[1]{\textcolor{red}{#1}}
\newcommand{\emphg}[1]{\textcolor{comment-green}{#1}}
\newcommand{\empho}[1]{\textcolor{orange}{#1}}
\newcommand{\emphb}[1]{\textcolor{blue}{#1}}
\newcommand{\comment}[1]{\textcolor{comment-green}{#1}}
\makeatletter
\@ifclassloaded{beamer}{}
{
  \newcommand{\alert}[1]{\textcolor{red}{\textbf{#1}}}
}
\makeatother
\newcommand{\greencomment}[1]{\comment{$\triangleright$ #1}}
\newcommand{\redcomment}[1]{\emphr{$\triangleright$ #1}}
\newcommand{\bluecomment}[1]{\emphb{$\triangleright$ #1}}
\newcommand{\alertcomment}[1]{\alert{$\triangleright$ #1}}

% Emphasis with box frame
%% Used in normal context
\newcommand{\alertbox}[1]{{%
    \setlength{\fboxrule}{.4mm}
    \color{red}\fbox{{\color{black}#1}}}}
\newcommand{\embox}[1]{{\setlength{\fboxrule}{.4mm}\fbox{#1}}}
%% Used in math context
\newcommand{\alertboxm}[1]{{%
    \setlength{\fboxrule}{.4mm}
    \color{red}\boxed{{\color{black}#1}}}}
\newcommand{\emboxm}[1]{{\setlength{\fboxrule}{.4mm}\boxed{#1}}}

% % Automatically hyphenate within \texttt.
% % Ref: https://tex.stackexchange.com/questions/44361/how-to-automatically-hyphenate-within-texttt
% \makeatletter
% \DeclareRobustCommand\ttfamily
%         {\not@math@alphabet\ttfamily\mathtt
%          \fontfamily\ttdefault\selectfont\hyphenchar\font=-1\relax}
% \makeatother
% \DeclareTextFontCommand{\mytexttt}{\ttfamily\hyphenchar\font=`\_\relax}
\let\oldtexttt\texttt
\renewcommand{\texttt}[1]{\oldtexttt{\hl{#1}}}

%%%%%%%%%%%%%%%%%%%%%%%%%%%%%%%%%%%%%%%
% Self-defined miscellaneous commands
%%%%%%%%%%%%%%%%%%%%%%%%%%%%%%%%%%%%%%%
% et al
\newcommand{\etal}{{\it et al}}

% Creative commons license
\newcommand{\cclic}{This work is licensed under a \href{http://creativecommons.org/licenses/by-nc-sa/4.0/}{Creative Commons Attribution-NonCommercial-ShareAlike 4.0 International License}. Copyright \copyright\ \the\year\ \href{mailto:jihuan_tian@hotmail.com}{Jihuan Tian}. All rights reserved.}
\newcommand{\cclicy}[1]{This work is licensed under a \href{http://creativecommons.org/licenses/by-nc-sa/4.0/}{Creative Commons Attribution-NonCommercial-ShareAlike 4.0 International License}. Copyright \copyright\ #1\ \href{mailto:jihuan_tian@hotmail.com}{Jihuan Tian}. All rights reserved.}

% Email
\newcommand{\myemail}{\href{mailto:jihuan_tian@hotmail.com}{jihuan\_tian@hotmail.com}}