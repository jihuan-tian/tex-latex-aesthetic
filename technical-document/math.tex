%%
%% Copyright (C) 2015 Jihuan Tian <jihuan_tian@hotmail.com>
%%  
%% This program is free software; you can redistribute it and/or modify
%% it under the terms of the GNU General Public License as published by
%% the Free Software Foundation; either version 2 of the License, or
%% (at your option) any later version.
%%  
%% This program is distributed in the hope that it will be useful,
%% but WITHOUT ANY WARRANTY; without even the implied warranty of
%% MERCHANTABILITY or FITNESS FOR A PARTICULAR PURPOSE.  See the
%% GNU General Public License for more details.
%%  
%% You should have received a copy of the GNU General Public License
%% along with this program; if not, write to the Free Software
%% Foundation, Inc., 59 Temple Place - Suite 330, Boston, MA 02111-1307, USA.
%%  

% Framed box
\usepackage[framemethod=tikz]{mdframed}

% Miscellaneous math packages
\usepackage{amsmath, amssymb, amstext, amsfonts, amscd, amsxtra}
\usepackage{mathrsfs}
% dashed lines in array
\usepackage{arydshln}
% For typesetting some mathematical symbols, such as := (\coloneqq),
% =: (\eqqcolon), etc.
\usepackage{mathtools}

% Specify math font
%% Use Concrete Italic + Concmath symbols (MetaFont) for math
\usepackage{concmath}
% For typesetting matrices with dashed separation lines
\usepackage{pmat}
%% Enlarge the default number of allowed columns in matrix environment
\setcounter{MaxMatrixCols}{20}

% For tikz drawing
\usepackage{tikz}
%% For drawing tikz commutative diagram
\usepackage{tikz-cd}
\usetikzlibrary{matrix, arrows, decorations.pathmorphing}

%%%%%%%%%%%%%%%%%%%%%%%%%%%%%%%%%%
% Self-defined math definitions
%%%%%%%%%%%%%%%%%%%%%%%%%%%%%%%%%%

% Math symbol commands
\newcommand{\intd}{\,{\rm d}}   % Symbol 'd' used in integration, such as 'dx'
\newcommand{\diff}{{\rm d}}     % Symbol 'd' used in differentiation
\newcommand{\Diff}{{\rm D}}     % Symbol 'D' used in differentiation
\newcommand{\pdiff}{\partial}   % Partial derivative
\newcommand{\DD}[2]{\frac{\diff}{\diff #2}\left( #1 \right)}
\newcommand{\Dd}[2]{\frac{\diff #1}{\diff #2}}
\newcommand{\PD}[2]{\frac{\pdiff}{\pdiff #2}\left( #1 \right)}
\newcommand{\Pd}[2]{\frac{\pdiff #1}{\pdiff #2}}
\newcommand{\rme}{{\rm e}}      % Exponential e
\newcommand{\rmi}{{\rm i}}      % Imaginary unit i
\newcommand{\rmj}{{\rm j}}      % Imaginary unit j
\newcommand{\vect}[1]{\boldsymbol{#1}}       % Vector typeset in bold and italic
\newcommand{\normvect}{\vect{n}} % Normal vector: n
\newcommand{\dform}[1]{\overset{\rightharpoonup}{\boldsymbol{#1}}}       % Vector for differential form
\newcommand{\cochain}[1]{\overset{\rightharpoonup}{#1}}       % Vector for cochain
\newcommand{\Abs}[1]{\big\lvert#1\big\rvert} % Absolute value (single big vertical bar)
\newcommand{\abs}[1]{\lvert#1\rvert} % Absolute value (single vertical bar)
\newcommand{\Norm}[1]{\big\lVert#1\big\rVert} % Norm (double big vertical bar)
\newcommand{\norm}[1]{\lVert#1\rVert} % Norm (double vertical bar)
\newcommand{\ouset}[3]{\overset{#3}{\underset{#2}{#1}}} % over and under set
% Super/subscript for column index of a matrix, which is used in tensor analysis.
\newcommand{\cscript}[1]{\;\; #1}
\newcommand{\suchthat}{\textit{S.T.\;}} % S.T., such that
% Star symbol used as prefix in front of a paragraph with no indent
\newcommand{\prefstar}{\noindent$\ast$ }      
% Big vertical line restricting the function.
% Example: $u(x)\restrict_{\Omega_0}$
\newcommand{\restrict}{\big\vert}

% Math operators which are typeset in Roman font
\DeclareMathOperator{\sgn}{sgn} % Sign function
\DeclareMathOperator{\erf}{erf} % Error function
\DeclareMathOperator{\Bd}{Bd}   % Boundary of a set or domain, used in topology
\DeclareMathOperator{\Int}{Int} % Interior of a set or domain, used in topology
\DeclareMathOperator{\rank}{rank} % Rank of a matrix
\DeclareMathOperator{\divergence}{div} % Divergence
\DeclareMathOperator{\curl}{curl} % Curl
\DeclareMathOperator{\grad}{grad} % Gradient
\DeclareMathOperator{\diag}{diag} % Diagonal
\DeclareMathOperator{\tr}{tr} % Trace
\DeclareMathOperator{\lhs}{LHS} % Left hand side
\DeclareMathOperator{\rhs}{RHS} % Right hand side
\DeclareMathOperator{\argmax}{argmax}
\DeclareMathOperator{\argmin}{argmin}

\mathchardef\ordinarycolon\mathcode`\:
\mathcode`\:=\string"8000
\begingroup \catcode`\:=\active
  \gdef:{\mathrel{\mathop\ordinarycolon}}
\endgroup

% Mathematical environment
%% standard: enable the standard set of theorems.
%% thmmarks: enable the automatic placement of endmarks, which should be used
%% together with the option amsmath when the amsmath package has been loaded.
%% thref: enables the extended reference features.
\usepackage{framed}
\usepackage[standard, framed, amsmath, hyperref, thmmarks, thref]{ntheorem}
% Define new mathematical environment.
\newtheorem{mynote}{Note}
\newtheorem{mycomment}{Comment}

% %%%%%%%%%%%%%%%%%%%%%%%%%%%%
% Pseudocode typesetting
% %%%%%%%%%%%%%%%%%%%%%%%%%%%%
% Use "algorithm" environment for floating an algorithm.
\usepackage{algorithm}
% Use algorithmicx package
\usepackage{algpseudocode}

% Draw vertical lines to visualize the hierarchy.
\makeatletter
% This is the vertical rule that is inserted
% \def\therule{\makebox[\algorithmicindent][l]{\hspace*{.5em}\vrule height
%     1\baselineskip depth .25\baselineskip}}%
\def\therule{\makebox[\algorithmicindent][l]{\hspace*{.5em} \vrule height 0.75 \baselineskip depth .25\baselineskip}}%

\newtoks\therules% Contains rules
\therules={}% Start with empty token list
\def\appendto#1#2{\expandafter#1\expandafter{\the#1#2}}% Append to token list
\def\gobblefirst#1{% Remove (first) from token list
  #1\expandafter\expandafter\expandafter{\expandafter\@gobble\the#1}}%
\def\LState{\State\unskip\the\therules}% New line-state
\def\pushindent{\appendto\therules\therule}%
\def\popindent{\gobblefirst\therules}%
\def\printindent{\unskip\the\therules}%
\def\printandpush{\printindent\pushindent}%
\def\popandprint{\popindent\printindent}%

%      ***      DECLARED LOOPS      ***
% (from algpseudocode.sty)
\algdef{SE}[WHILE]{While}{EndWhile}[1]
  {\printandpush\algorithmicwhile\ #1\ \algorithmicdo}
  {\popandprint\algorithmicend\ \algorithmicwhile}%
\algdef{SE}[FOR]{For}{EndFor}[1]
  {\printandpush\algorithmicfor\ #1\ \algorithmicdo}
  {\popandprint\algorithmicend\ \algorithmicfor}%
\algdef{S}[FOR]{ForAll}[1]
  {\printindent\algorithmicforall\ #1\ \algorithmicdo}%
\algdef{SE}[LOOP]{Loop}{EndLoop}
  {\printandpush\algorithmicloop}
  {\popandprint\algorithmicend\ \algorithmicloop}%
\algdef{SE}[REPEAT]{Repeat}{Until}
  {\printandpush\algorithmicrepeat}[1]
  {\popandprint\algorithmicuntil\ #1}%
\algdef{SE}[IF]{If}{EndIf}[1]
  {\printandpush\algorithmicif\ #1\ \algorithmicthen}
  {\popandprint\algorithmicend\ \algorithmicif}%
\algdef{C}[IF]{IF}{ElsIf}[1]
  {\popandprint\pushindent\algorithmicelse\ \algorithmicif\ #1\ \algorithmicthen}%
\algdef{Ce}[ELSE]{IF}{Else}{EndIf}
  {\popandprint\pushindent\algorithmicelse}%
\algdef{SE}[PROCEDURE]{Procedure}{EndProcedure}[2]
   {\printandpush\algorithmicprocedure\ \textproc{#1}\ifthenelse{\equal{#2}{}}{}{(#2)}}%
   {\popandprint\algorithmicend\ \algorithmicprocedure}%
\algdef{SE}[FUNCTION]{Function}{EndFunction}[2]
   {\printandpush\algorithmicfunction\ \textproc{#1}\ifthenelse{\equal{#2}{}}{}{(#2)}}%
   {\popandprint\algorithmicend\ \algorithmicfunction}%
\makeatother

% Redefine the comment used in algorithmx.
\algrenewcommand{\algorithmiccomment}[1]{\comment{$\triangleright$ #1}}

\makeatletter
\newenvironment{breakablealgorithm}
  {% \begin{breakablealgorithm}
   \begin{center}
     \refstepcounter{algorithm}% New algorithm
     \hrule height.8pt depth0pt \kern2pt% \@fs@pre for \@fs@ruled
     \renewcommand{\caption}[2][\relax]{% Make a new \caption
       {\raggedright\textbf{\fname@algorithm~\thealgorithm} ##2\par}%
       \ifx\relax##1\relax % #1 is \relax
         \addcontentsline{loa}{algorithm}{\protect\numberline{\thealgorithm}##2}%
       \else % #1 is not \relax
         \addcontentsline{loa}{algorithm}{\protect\numberline{\thealgorithm}##1}%
       \fi
       \kern2pt\hrule\kern2pt
     }
  }{% \end{breakablealgorithm}
     \kern2pt\hrule\relax% \@fs@post for \@fs@ruled
   \end{center}
  }
\makeatother
